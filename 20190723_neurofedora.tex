%% 2019 NeuroFedora contributors

%% packages %%
% support for coloured text
\usepackage{xcolor}
\definecolor{FedoraBlue}{cmyk}{1.0,0.46,0.0,0.0}
\definecolor{FedoraDarkBlue}{cmyk}{1.0,0.57,0.0,0.38}
\definecolor{FriendsMagenta}{cmyk}{0.0,0.8,0.4,0.0}
\definecolor{FeaturesOrange}{cmyk}{0.0,0.5,1.0,0.0}
\definecolor{FirstGreen}{cmyk}{0.5,0.0,1.0,0.0}
\definecolor{FreedomPurple}{cmyk}{0.57,0.46,0.0,0.0}

% IPA
\usepackage{tipa}
\usepackage[scale=2]{ccicons}
\usepackage{amssymb}
\usepackage{tikz}
\usetikzlibrary{arrows.meta, arrows, positioning}
\usepackage{jneurosci}
\usepackage{subfig}
\usepackage[T1]{fontenc}
\usepackage[utf8]{inputenc}
\usepackage[style=verbose,backend=biber,autocite=footnote]{biblatex}
\addbibresource{/home/asinha/Documents/01_Readables/00_research_papers/masterbib.bib}
% Use opensans
\usepackage[default,osfigures,scale=0.95]{opensans}
% for strike through
\usepackage[normalem]{ulem}
% links, urls, refs
\usepackage{hyperref}
\hypersetup{colorlinks,linkcolor=FreedomPurple,urlcolor=FreedomPurple}
% graphics
\usepackage{graphicx}
% algorithm
\usepackage{algorithmic}
\usepackage{textcomp}
\usepackage{wrapfig}
\usepackage{textgreek}
\usepackage{euler}

% beamer theme
\usetheme[numbering=fraction]{metropolis}
\usefonttheme[onlymath]{serif}
\setbeamerfont{footnote}{size=\tiny}
\setbeamerfont{caption}{size=\tiny}
\setbeamercolor{alerted text}{fg=FeaturesOrange}
\setbeamercolor{progress bar}{fg=FriendsMagenta}
\setbeamercolor{title separator}{fg=FriendsMagenta}
\setbeamercolor{frametitle}{bg=FedoraDarkBlue}

% Not needed in metropolis, but in general footnote citation fixes: https://tex.stackexchange.com/questions/44217/how-can-i-stop-footcite-from-hijacking-my-beamer-columns
% how to use multiple references to the same footnote: https://tex.stackexchange.com/questions/27763/beamer-multiple-references-to-the-same-footnote

%% title %%
\title[NeuroFedora]{\includegraphics[keepaspectratio,width=.25\textwidth]{images/NeuroFedoraLogo01.png}\\NeuroFedora}
\subtitle{FOSS and Free/Open (neuro) Science}
\author{NeuroFedora contributors}
\date[]{}
%% document begins %%
\begin{document}

% title frame %%
\begin{frame}
  \titlepage{}
\end{frame}

%% Three slides for 5 minutes, so 25--30 for 50 minutes.

%% First we define the problem statement. What is it that neuroscience is trying to achieve?
%% Added advantage is that this bit has lots of nice pictures, so it helps to gain the audience's attention.
\section{Problem statement: the brain}
\begin{frame}[c]{The brain: neurons}
  \begin{figure}[h]
    \centering
    \includegraphics[width=\linewidth]{images/Neurons.jpg}
  \end{figure}
  \note[item]{The brain is composed of specialised cells that enable it to process information by the use of electrical impulses}
  \note[item]{As the figure shows, these cells, neurons, have specialised into many many types. They serve different functions, include different proteins and markers, and can be classified in many different ways.}
\end{frame}
\begin{frame}[c]{The brain: in numbers: neurons}
  \begin{columns}
    \begin{column}{0.5\textwidth}
      \begin{figure}[h]
        \centering
        \includegraphics[width=\textwidth]{images/brain-sizes.jpg}
      \end{figure}
    \end{column}
    \begin{column}{0.5\textwidth}
      \begin{itemize}
        \item \alert{86B} neurons\footnotemark{}.
      \end{itemize}
    \end{column}
  \end{columns}
  \vspace{0.2cm}
  \footnotetext[1]{\fullcite{Herculano-Houzel2009}}
  \note[item]{The most recent estimate puts the number of neurons in the human brain at 86B.}
\end{frame}
\begin{frame}[c]{The brain: in numbers: synapses}
  \begin{columns}
    \begin{column}{0.5\textwidth}
      \begin{figure}[h]
        \centering
        \includegraphics[width=\textwidth]{images/reconstruction.jpg}
      \end{figure}
    \end{column}
    \begin{column}{0.5\textwidth}
      \begin{itemize}
        \item \alert{Thousands} of connections between neurons \alert{(synapses)}\footnotemark[2].
        \item Synapses are also of different types, and serve different functions.
          \pause{}
        \item Synapses underlie \alert{learning}\footnotemark[3].
      \end{itemize}
    \end{column}
  \end{columns}
  \vspace{0.2cm}
  \footnotetext[2]{\href{https://drexel.edu/medicine/about/departments/neurobiology-anatomy/research/gao-lab/images/}{Image from The Gao lab, College of Medicine, Drexel University.}}
  \footnotetext[3]{\fullcite{Hebb1949}}
  \note[item]{Each neuron connects with thousands of other neurons, forming a massive network.}
  \note[item]{So, the brain can be thought of as a massively parallel processor.}
\end{frame}
\begin{frame}[c]{So, we want to know (among other things)}
  \begin{itemize}
    \item how the brain functions (\alert{physiology}),
    \item how it is structured (\alert{anatomy}),
    \item about its chemicals (\alert{pharmacology, biochemistry}),
    \item \ldots{}
      \pause{}
    \item how it processes information (\alert{computational}),
    \item about behaviours, and cognition (\alert{behavioural, cognitive}),
    \item \ldots{}
  \end{itemize}
\end{frame}
\begin{frame}[c]{with the aim of applying this knowledge to}
  \begin{itemize}
    \item \alert{disease} prevention and treatment,
    \item \ldots{}
      \pause{}
    \item brain inspired \alert{computing},
    \item \ldots{}
      \pause{}
    \item philosophy and consciousness,
  \end{itemize}
  \note[item]{To take applications from the extreme ends of the spectrum: immediate clinical applications, immediate technological applications.}
\end{frame}
\section{Research pipeline}
\begin{frame}[c]{General workflow}
  \begin{figure}[h]
    \centering
    \only<1>{\input{images/Neuroscience-cycle}}%
    \only<2>{\input{images/Neuroscience-cycle-complex}}
    \note[item]{A simplified diagram. Actually a lot more complex}
  \end{figure}
\end{frame}
\begin{frame}[c]{Tools of the trade}
  \begin{itemize}
    \item \textcolor{FedoraBlue}{Experimental:}
      \begin{itemize}
        \item EEG, ECoG, intracellular and extracellular single and multi neuron recording,
        \item CT, DOI, MRI, f-MRI, MEG, PET,
      \end{itemize}
      \pause{}
    \item \textcolor{FriendsMagenta}{Data analysis:}
      \begin{itemize}
        \item Statistics,
        \item Machine Learning, Big Data, Deep learning,
      \end{itemize}
      \pause{}
    \item \textcolor{FeaturesOrange}{Theory} and \textcolor{FirstGreen}{modelling:}
      \begin{itemize}
        \item Simulators of all kinds, 
      \end{itemize}
      \pause{}
    \item \textcolor{FedoraDarkBlue}{Dissemination of results\footnotemark[4].}
  \end{itemize}
  \note[item]{Lots of hardware and software is required for basic neuroscience research.}
  \footnotetext[4]{To a non-specialist audience.}
\end{frame}
\begin{frame}[c]{FOSS and Free/Open Science share ideals}
  FOSS\@:\\\alert{Everyone} should have the freedom to \alert{share, study, and modify} software\footnotemark[5].\\
  \pause{}
  \vspace{0.5cm}
  Free/Open science:\\\alert{Everyone} should have the freedom to \alert{share, study, and modify} scientific material.\\
  \pause{}
  \vspace{0.5cm}
  \alert{Free/Open Science implicitly includes FOSS.}
  \footnotetext[5]{\href{https://u.fsf.org/user-liberation}{Free software foundation}}
\end{frame}
\begin{frame}[c]{FOSS and Open Science are catching on:}
  \begin{itemize}
    \item There are now active efforts to:
      \begin{itemize}
        \item use \alert{FOSS\footnotemark[6]},
        \item standardise open access publishing,
        \item use \alert{open formats} for data,
      \end{itemize}
  \end{itemize}
  \footnotetext[6]{\href{http://opensourceforneuroscience.org/}{Open source for neuroscience}}
\end{frame}
\section{Fedora and Open Science?}
\begin{frame}[c]{User/developer community}
  \begin{itemize}
    \item \alert{various specialities:} biologists, mathematicians, physicists, chemists, psychologists, \ldots, 
      \pause{}
    \item \alert{small proportion of trained software developers},
  \end{itemize}
\end{frame}
\begin{frame}[c]{Anecdotal observations on software}
  \begin{itemize}
    \item often single developer, small user-developer communities,
    \item limited code quality,
    \item limited use of established best practices,
    \item limited testing for correctness (!),
    \item limited maintenance,
    \item often limited lifespan,
  \end{itemize}
  \note[item]{Give how interdisciplinary neuroscience is, most researchers are NOT trained in development}
  \note[item]{This implies, and this is based on anecdotal evidence, that the software used in research is not of the best quality}
  \note[item]{The silver lining is that a lot of it is FOSS, and more and more is becoming FOSS by default.}
\end{frame}
\begin{frame}[c]{Free/Open Science}
  \begin{itemize}
    \item FOSS for 
  \end{itemize}
\end{frame}
\end{document}
